\documentclass[protokol.tex]{subfiles}
\begin{document}
\subsection*{Měření modulu $E$ z protažení drátu}
Při působení síly F na drát průřezu S se drát pružnou deformací  prodlouží o \cite{stud_text}
\begin{equation} \label{eq:delta_l}
\Delta l = \frac{1}{E}\frac{l_0 F}{S}.
\end{equation}
$E$ je modul pružnosti v tahu, \cite{stud_text}
\begin{equation} \label{eq:modul_dratu}
E = \frac{\sigma}{\epsilon} = \frac{l_0 F}{\Delta l S} = \frac{4 l_0 F}{\Delta l \pi d^2}
\end{equation}

Prodloužení drátu se měří zrcátkovou metodou. Protažení drátu se převádí na pootočení $\Delta \alpha$ zrcátka upevněného na ose kladky s poloměrem $r$. \cite{stud_text}
\begin{equation} \label{eq:alpha}
\Delta l = r \Delta \alpha
\end{equation}

Ve vzdálenosti $L$ od zrcátka je umístěna svislá stupnice, před otočením zrcátka je v dalekohledu vidět dílek stupnice $n_0$, po otočení dílek $n$. Pro malé úhly pootočení platí pro prodloužení drátu přibližný vztah \cite{stud_text}
\begin{equation} \label{eq:delta_l_2}
\Delta l \approx \frac{r (n_0 - n)}{2 L}
\end{equation}

\subsection*{Měření modulu $E$ z průhybu trámku}
Při zatěžování vodorovného kovového trámku, podepřeného dvěma břity ve vzdálenosti $l$, silou $F$ se trámek prohne průhybem 
\begin{equation}
y = \frac{F l^3}{48 E I_p},
\end{equation}
kde $I_p$ je plošný moment setrvačnosti průřezové plochy tyče vzhledem k vodorovné ose, kolmé k délce trámku a procházející těžištěm. Pro obdélníkový průřez trámku výšky $b$ a šířky $a$ lze $I_p$ vyjádřit vztahem
\begin{equation}
I_p = \frac{ab^3}{12}.
\end{equation}

Modul pružnosti poté dostaneme jako 
\begin{equation} \label{eq:modul_prohnuti}
E = \frac{F l^3}{4 y a b^3}.
\end{equation}

\subsection*{Statistické vyhodnocení}
Průměrná hodnota naměřených veličin při $n$ měřeních je počítána podle vzorce aritmetického průměru 
\cite{cizek_10}
$$ \overline{x} = \frac{1}{n} \sum\limits_{i=1}^n{x_i}.$$
Statistická chyba $\sigma_{stat}$ aritmetického průměru se získá ze vztahu \cite{cizek_10}
$$ \sigma_{stat} = \frac{\sqrt{\frac{1}{n-1} \sum\limits_{i=1}^n{(x_i - \overline{x})^2}}}{\sqrt{n}}. $$
Absolutní chyba je potom získána z $\sigma_{stat}$ a chyby měřidla $\sigma_{\text{měř}}$ jako \cite{cizek_1}
$$ \sigma_{abs} = \sqrt{\sigma_{\text{\textit{měř}}}^2 + \sigma_{stat}^2}$$

Chyba výpočtů se řídí zákonem přenosu chyb \cite{cizek_9}, lineární regrese podle metody nejmenších čtverců 
\cite{cizek_11}.

\subsection*{Pomůcky}
Posuvné měřidlo, pásové měřidlo, drát, kladka, zrcátko, stupnice, dalekohled, závaží, břity, kovové trámky, objektivový mikrometr
\end{document}
