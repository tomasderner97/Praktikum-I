\documentclass[11pt]{article}
\usepackage[utf8]{inputenc}
\usepackage[czech]{babel}

\usepackage[margin=3cm]{geometry}
\usepackage{fancyhdr}
\usepackage[decimalsymbol=comma]{siunitx}
\usepackage{float}
\usepackage{textcomp}
\usepackage{amsmath}
\usepackage{hyperref}

\pagestyle{fancy}
\fancyhead{}

\fancyhead[L]{Otáčení tuhého tělesa}
\fancyhead[C]{Fyzikální praktikum I}
\fancyhead[R]{Tomáš Derner}

\parindent 0ex

\begin{document}

\section*{Pracovní úkol}
\begin{enumerate}
\item Změřte momenty setrvačnosti kvádru vzhledem k hlavním osám setrvačnosti.
\item Určete složky jednotkového vektoru ve směru zadané obecné osy rotace kvádru v souřadné soustavě dané hlavními osami setrvačnosti.
\item Vypočítejte moment setrvačnosti kvádru vzhledem k zadané obecné ose rotace. Výsledek ověřte měřením.
\item Měrně ověřte Steinerovu větu.
\end{enumerate}

\section*{Teorie}
\subsection*{Metoda torzních kmitů}
Metoda torzních kmitů se využívá k měření momentu setrvačnosti tělesa vzhledem k ose procházející jeho hmotným středem. Periodu takových kmitů vyjádříme jako 
\begin{equation} \label{eq:T_torz}
T = 2\pi \sqrt{\frac{I}{D}},
\end{equation}
kde $I$ je právě moment setrvačnosti a $D$ direkční moment vlákna. 
Direkční moment lze eliminovat použitím této metody na těleso se známým momentem setrvačnosti,
\begin{equation} \label{eq:T_torz_znamy}
T_v = 2\pi \sqrt{\frac{I_v}{D}}.
\end{equation}
V našem případě je známým tělesem válec, jehož moment setrvačnosti vypočteme z jeho hmotnosti $m_v$ a poloměru $r$ takto:
\begin{equation} \label{eq:I_valce}
I_v = \frac{1}{2} m r^2.
\end{equation}

Kombinací vztahů \eqref{eq:T_torz}, \eqref{eq:T_torz_znamy} a \eqref{eq:I_valce} dostaneme 
\begin{equation} \label{eq:I_kvadru}
I = \frac{T^2}{{T_v}^2}I_v
\end{equation}
pro $T$ periodu kmitů tělesa, zavěšeného v bodě osy, vzhledem ke které stanovujeme moment setrvačnosti, $T_v$ periodu kmitů válce a $I_v$ jeho moment setrvačnosti vůči ose kolmé na podstavu a procházející hmotným středem. 

Pokud není překročena mez úměrnosti vlákna, perioda nezávisí na výchylce, ta může být až 90\textdegree.
Touto metodou změříme hlavní složky tenzoru setrvačnosti kvádru, tedy momenty setrvačnosti vůči osám kolmým na jeho stěny. Ty následně využijeme v dalších výpečtech.

\subsection*{Měření setrvačnosti vzhledem k obecné ose procházející tělesem}
Chceme-li měřit moment setrvačnosti vůči obecné ose, musíme znát složky jednotkového vektoru ve směru této osy v souřadné soustavě dané hlavními osami setrvačnosti. V našem případě se zajímáme o moment setrvačnosti vůči prostorové úhlopříčce kvádru, jednotkový vektor $\nu = (\nu_x, \nu_y, \nu_z)$ vyjádříme takto:
\begin{equation} \label{eq:slozky}
\nu_x = \frac{a}{\sqrt{a^2 + b^2 + c^2}},\ \ \ 
\nu_y = \frac{b}{\sqrt{a^2 + b^2 + c^2}},\ \ \ 
\nu_z = \frac{c}{\sqrt{a^2 + b^2 + c^2}},
\end{equation} 
přičemž $a$, $b$ a $c$ jsou strany kvádru ($a > b > c$) a souřadnice $x$ má směr hrany $a$, $y$ směr hrany $b$ a $z$ směr hrany $c$.
Moment setrvačnosti pak dostaneme jako 
\begin{equation} \label{eq:I_uhlopricky}
I_u = \nu_x^2 I_x + \nu_y^2 I_y + \nu_z^2 I_z.
\end{equation}
Dosaženou hodnotu ověříme měřením.

\subsection*{Steinerova věta}
Steinerova věta umožňuje vypočítat moment setrvačnosti vůči ose, která neprochází hmotným středem tělesa. Jestliže $I$ je hledaný moment setrvačnosti, $I_0$ moment setrvačnosti vůči ose kolmé na zadanou osu procházející hmotným středem, $m$ hmotnost tělesa a $d$ vzdálenost obou os, můžeme psát
\begin{equation} \label{eq:Steiner}
I = I_0 + m d^2.
\end{equation}

Zavěsíme-li tyč na jejím konci, vytvoříme fyzické kyvadlo, jehož perioda bude pro moment setrvačnosti $I_k$ vůči ose, kolem které tyč kýve, hmotnost tyče $m$, tíhové zrychlení $g$ a vzdálenost osy od hmotného středu $d$
\begin{equation} \label{eq:kyvadlo}
T = 2 \pi \sqrt{\frac{I_k}{mgd}}.
\end{equation}
Steinerovu větu ověříme tak, že změříme moment setrvačnosti tyče vůči ose procházející hmotným středem pomocí torzních kmitů a porovnáme s hodnotou, která je na metodě torzních kmitů nezávislá. Získáme ji tak, že pomocí \eqref{eq:kyvadlo} vypočteme moment setrvačnosti vůči ose, kolem které tyč kýve, a přepočítáme právě pomocí Steinerovy věty. Z \eqref{eq:kyvadlo} vyjádříme 
\begin{equation}
I_k = \frac{T^2 mgd}{4\pi} ,
\end{equation}
potom z \eqref{eq:Steiner}
\begin{equation} \label{eq:I_tyc_kyv}
I_tk = \frac{T^2 mgd}{4\pi} - m d^2.
\end{equation}

\subsection*{Statistické vyhodnocení}
Průměrná hodnota naměřených veličin při $n$ měřeních je počítána podle vzorce aritmetického průměru 
$$ \overline{x} = \frac{1}{n} \sum\limits_{i=1}^n{x_i}.$$
Statistická chyba $\sigma_{stat}$ aritmetického průměru se získá ze vztahu
$$ \sigma_{stat} = \frac{\sqrt{\frac{1}{n-1} \sum\limits_{i=1}^n{(x_i - \overline{x})^2}}}{\sqrt{n}}. $$
Absolutní chyba je potom získána z $\sigma_{stat}$ a systematické chyby $\sigma_{sys}$ jako
$$ \sigma_{abs} = \sqrt{\sigma_{sys}^2 + \sigma_{stat}^2}$$

Chyba výpočtů se řídí zákonem přenosu chyb.

\subsection*{Pomůcky}
Kvádr a válec se závity, tyč se závitem a břity, drát se závitem, pásové měřidlo, posuvné měřidlo, mikrometr, váhy, stopky, lůžko pro kyvadlo, upínací zařízení
\section*{Výsledky měření}
Podmínky měření jsou uvedeny v následující tabulce.

\begin{table}[H]	% Podmínky
\centering
\begin{tabular}{|c|c|c|}
\hline
Teplota [\textdegree C] & Tlak [hPa]  & vlhkost [\% RH] \\ 
\hline
$24.6 \pm 0.4$ & $979.7 \pm 2$ & $23.4 \pm 2.5$ \\ 
\hline
\end{tabular}
\caption{Podmínky měření}
\label{tab:podminky}
\end{table}

V tabulkách \ref{tab:kvadr}, \ref{tab:valec} a \ref{tab:tyc} jsou uvedeny rozměry kvádru, válce a tyče, jejich průměrné hodnoty, systematické, statistické a absolutní chyby. Absolutní chyby jsou počítány jako odmocnina ze čtverce systematické a statistické chyby. Všechny délkové rozměry kromě $ 2d $ byly měřeny posuvným měřidlem, $ 2d $ pásovým měřidlem a hmotnosti na digitálních váhách.

\begin{table}[H]	% Rozměry kvádru
\centering
\begin{tabular}{|c|c|c|c|c|c|c|c|c|c|} \hline

měření	& 1		& 2		& 3		& 4		& 5		
& průměr & $ \sigma_{sys} $ & $ \sigma_{stat} $ & $ \sigma_{abs} $ \\ 
\hline
$a$ [mm]& 127.82 & 127.84 & 127.82 & 127.80 & 127.80 & 127.82 & 0.03 & 0.017 & 0.03 \\ 
\hline
$b$ [mm]& 63.96 & 63.97 & 63.98 & 63.93 & 63.93 & 63.95 & 0.03 & 0.023 & 0.04 \\ 
\hline
$c$	[mm]& 18.97 & 18.97 & 19.01 & 18.97 & 19.01 & 18.99 & 0.03 & 0.022 & 0.04 \\ 
\hline
$m_k$[g]& 1071.5 & 1071.5 & 1071.5 & 1071.5 & 1071.5 & 1071.5 & 0.1 & 0.0 & 0.1 \\ 
\hline

\end{tabular}
\caption{Rozměry kvádru}
\label{tab:kvadr}
\end{table}


\begin{table}[H]	% Rozměry válce
\centering
\begin{tabular}{|c|c|c|c|c|c|c|c|c|c|} \hline

měření	& 1		& 2		& 3		& 4		& 5		
& průměr & $ \sigma_{sys} $ & $ \sigma_{stat} $ & $ \sigma_{abs} $ \\ 
%\hline
%$v$ 	[mm]& 12.66 & 12.66 & 12.66 & 12.68 & 12.68 & 12.67 & 0.01 & 0.011 & 0.01 \\ 
\hline
$2r$ 	[mm]& 107.99 & 108.02 & 107.95 & 107.98 & 107.99 & 107.99 & 0.03 & 0.025 & 0.04 \\
\hline
$r$		[mm]& 54.00 & 54.01 & 53.98 & 53.99 & 54.00 & 54.00 & 0.015 & 0.011 & 0.02 \\
\hline
$m_v$ 	[g] & 903.6 & 903.6 & 903.6 & 903.6 & 903.6 & 903.6 & 0.1 & 0 & 0.1 \\ 
\hline

\end{tabular}
\caption{Rozměry válce}
\label{tab:valec}
\end{table}


\begin{table}[H]	% Rozměry tyče
\centering
\begin{tabular}{|c|c|c|c|c|c|c|c|c|c|} \hline

měření	& 1		& 2		& 3		& 4		& 5		
& průměr & $ \sigma_{sys} $ & $ \sigma_{stat} $ & $ \sigma_{abs} $ \\ 
\hline
$2d$  [mm]& 314 & 313 & 313 & 314 & 313 & 313 & 1 & 0.5 & 1 \\
\hline
$d$   [mm]& 157.0 & 156.5 & 156.5 & 157.0 & 156.5 & 156.7 & 0.5 & 0.3 & 0.6 \\
\hline
$m_t$ [g] & 281.3 & 281.4 & 281.4 & 281.4 & 281.4 & 281.4 & 0.1 & 0.05 & 0.1 \\
\hline

\end{tabular}
\caption{Rozměry tyče}
\label{tab:tyc}
\end{table}

\newpage
V tabulce \ref{tab:t_kvadr} jsou uvedeny periody torzních kmitů měřené na stopkách s ručním spouštěním, jako systematická chyba měření byla zvolena standardní hodnota $0.2 s$. Absolutní chyba je počítána stejně jako výše. Jako dobrý kompromis mezi přesností a časovou náročností bylo zvoleno 5 kmitů na jedno měření.

\begin{table}[H]	% Periody kvádru
\centering
\begin{tabular}{|c|c|c|c|c|c|c|c|c|c|} \hline

měření	& 1		& 2		& 3		& 4		& 5		
& průměr & $ \sigma_{sys} $ & $ \sigma_{stat} $ & $ \sigma_{abs} $ \\ 
\hline
$5T_x$[s] & 33.67 & 33.54 & 33.57 & 33.66 & 33.71 & 33.63 & 0.20 & 0.07 & 0.21 \\
\hline
$5T_y$[s] & 61.20 & 61.02 & 61.65 & 61.17 & 60.94 & 61.16 & 0.20 & 0.19 & 0.28 \\
\hline
$5T_z$[s] & 68.32 & 68.41 & 68.39 & 68.23 & 68.39 & 68.35 & 0.20 & 0.07 & 0.21 \\
\hline
$5T_u$[s] & 41.15 & 41.13 & 41.05 & 40.97 & 41.28 & 41.12 & 0.20 & 0.12 & 0.23 \\
\hline

$T_x$[s] & 6.73 & 6.71 & 6.71 & 6.73 & 6.74 & 6.72 & 0.04 & 0.01 & 0.04 \\
\hline
$T_y$[s] & 12.24 & 12.20 & 12.29 & 12.23 & 12.19 & 12.23 & 0.04 & 0.03 & 0.05 \\
\hline
$T_z$[s] & 13.66 & 13.68 & 13.68 & 13.65 & 13.68 & 13.67 & 0.04 & 0.01 & 0.04 \\
\hline
$T_u$[s] & 8.23 & 8.23 & 8.21 & 8.19 & 8.26 & 8.22 & 0.04 & 0.03 & 0.05 \\
\hline

\end{tabular}
\caption{Periody kvádru}
\label{tab:t_kvadr}
\end{table}

Z důvodu přetržení drátu v průběhu měření a tutíž změně direkčního momentu bylo zapotřebí měřit periodu torzních kmitů válce zvlášť pro výpočty momentů setrvačnosti kvádru a tyče. V tabulce \ref{tab:t_valec_kvadr} jsou zaznamenány periody pro výpočty s kvádrem, měřené na týchž stopkách.

\begin{table}[H]	% Periody válce pro kvádr
\centering
\begin{tabular}{|c|c|c|c|c|c|c|c|c|c|} \hline

měření	& 1		& 2		& 3		& 4		& 5		
& průměr & $ \sigma_{sys} $ & $ \sigma_{stat} $ & $ \sigma_{abs} $ \\ 
\hline
$5T_{vk}$ [s]& 63.95 & 63.99 & 64.01 & 63.93 & 63.95 & 63.97 & 0.20 & 0.03 & 0.20 \\
\hline
$T_{vk}$ [s] & 12.79 & 12.80 & 12.80 & 12.79 & 12.79 & 12.79 & 0.04 & 0.01 & 0.04 \\
\hline

\end{tabular}
\caption{Periody válce pro kvádr}
\label{tab:t_valec_kvadr}
\end{table}

V tabulce \ref{tab:t_tyc_kyv} jsou uvedeny periody kmitů tyče jakožto fyzického kyvadla. Počáteční výchylka kyvadla byla asi 5\textdegree, původně bylo zvoleno 20 kmitů na jedno měření, z důvodu systematické chyby při počítání kmitů, odhalené v průběhu měření, bylo nutné tuto hodnotu upravit na 19 kmitů. Statistické vyhodnocení proběhlo stejně jako výše.

\begin{table}[H]	% Periody tyče jako kyvadla
\centering
\begin{tabular}{|c|c|c|c|c|c|c|c|c|c|} \hline

měření	& 1		& 2		& 3		& 4		& 5		
& průměr & $ \sigma_{sys} $ & $ \sigma_{stat} $ & $ \sigma_{abs} $ \\ 
\hline
$19T_{tk}$ [s]& 17.83 & 17.91 & 17.89 & 17.91 & 17.9 & 17.89 & 0.20 & 0.03 & 0.20 \\
\hline
$T_{vk}$ [s] & 0.94 & 0.94 & 0.94 & 0.94 & 0.94 & 0.94 & 0.01 & 0.00 & 0.01 \\
\hline

\end{tabular}
\caption{Periody tyče jako kyvadla}
\label{tab:t_tyc_kyv}
\end{table}

V tabulce \ref{tab:t_tyc_torz} jsou zaznamenány periody torzních kmitů tyče. Z časových důvodů bylo nezbytné omezit jak počet měření z pěti na tři, tak i počet kmitů na jedno měření z pěti na čtyři.

\begin{table}[H]	% Periody torzních kmitů tyče
\centering
\begin{tabular}{|c|c|c|c|c|c|c|c|} \hline

měření	& 1		& 2		& 3			
& průměr & $ \sigma_{sys} $ & $ \sigma_{stat} $ & $ \sigma_{abs} $ \\ 
\hline
$4T_{tt}$ [s]& 121.88 & 121.82 & 121.94 & 121.88 & 0.20 & 0.06 & 0.21 \\
\hline
$T_{tt}$ [s]& 30.47 & 30.46 & 30.49 & 30.47 & 0.05 & 0.02 & 0.05 \\
\hline

\end{tabular}
\caption{Periody torzních kmitů tyče}
\label{tab:t_tyc_torz}
\end{table}

Tabulka \ref{tab:t_valec_tyc} obsahuje periody torzních kmitů válce určených pro výpočet srovnávacího momentu setrvačnosti pro výpočty s tyčí z důvodu zmíněného defektu pomůcek. Počet měření je opět omezen na tři.

\begin{table}[H]	% Periody válce pro tyč
\centering
\begin{tabular}{|c|c|c|c|c|c|c|c|} \hline

měření	& 1		& 2		& 3			
& průměr & $ \sigma_{sys} $ & $ \sigma_{stat} $ & $ \sigma_{abs} $ \\ 
\hline
$4T_{vt}$ [s]& 84.44 & 84.32 & 84.16 & 84.31 & 0.20 & 0.14 & 0.24 \\
\hline
$T_{vt}$ [s]& 21.11 & 21.08 & 21.04 & 21.08 & 0.05 & 0.04 & 0.06 \\
\hline

\end{tabular}
\caption{Periody válce pro tyč}
\label{tab:t_valec_tyc}
\end{table}

\subsection*{Pracovní úkol 1}
Dosazením do \eqref{eq:I_valce} z tabulky \ref{tab:valec} dostáváme moment setrvačnosti válce 
$$ I_v = (1.317 \pm 0.003)\times \num{e-3} \ \si{\kilogram\metre\squared}. $$
Pomocí \eqref{eq:I_kvadru} a hodnot z tabulky \ref{tab:t_kvadr} a \ref{tab:t_valec_kvadr} získáme momenty setrvačnosti vůči hlavním osám kvádru
$$ I_x = (3.64 \pm 0.09)\times \num{e-4} \ \si{\kilogram\metre\squared}, $$
$$ I_y = (1.20 \pm 0.01)\times \num{e-3} \ \si{\kilogram\metre\squared}, $$
$$ I_z = (1.50 \pm 0.01)\times \num{e-3} \ \si{\kilogram\metre\squared}. $$

\subsection*{Pracovní úkol 2}
Pomocí \eqref{eq:slozky} s použitím hodnot z tabulky \ref{tab:kvadr} vypočítáme složky směrového vektoru úhlopříčky kvádru
$$ \nu_x = (886.5 \pm 0.5)\times \num{e-3}, $$
$$ \nu_y = (443.5 \pm 0.4)\times \num{e-3}, $$
$$ \nu_z = (131.7 \pm 0.1)\times \num{e-3}. $$

\subsection*{Pracovní úkol 3}
Použitím rovnice \eqref{eq:I_uhlopricky} dostaneme vypočítanou velikost momentu setrvačnosti vzhledem k úhlopříčce kvádru
$$ I_{uv} = (5.49 \pm 0.09)\times \num{e-4} \ \si{\kilogram\metre\squared}. $$
Dosazením hodnot z tabulky \ref{tab:t_kvadr} a \ref{tab:t_valec_kvadr} do rovnice \eqref{eq:I_kvadru} získáme hodnotu velikosti momentu setrvačnosti vzhledem k úhlopříčce kvádru obdrženou měřením
$$ I_{um} = (5.44 \pm 0.09)\times \num{e-4} \ \si{\kilogram\metre\squared}. $$

\subsection*{Pracovní úkol 4}
Moment setrvačnosti tyče vůči ose procházející hmotným středem kolmo na tyč dosažený přepočítáním momentu setrvačnosti vůči ose rovnoběžné s osou procházející hmotným středem pomocí \eqref{eq:I_tyc_kyv} a hodnot z tabulky \ref{tab:tyc} a \ref{tab:t_tyc_kyv} je
$$ I_{tk} = (2.75 \pm 0.17)\times \num{e-3} \ \si{\kilogram\metre\squared}. $$

Moment setrvačnosti tyče vůči ose procházející hmotným středem kolmo na tyč získaný metodou torzních kmitů obdržíme použitím rovnice \eqref{eq:I_kvadru} na hodnoty z tabulky \ref{tab:t_tyc_torz} a \ref{tab:t_valec_tyc}:
$$ I_{tt} = (2.78 \pm 0.22)\times \num{e-3} \ \si{\kilogram\metre\squared}. $$
\section*{Diskuse}
Největším zdrojem chyb při tomto experimentu bylo měření časových intervalů pomocí ručně spouštěných stopek, kde se projevila nízká reakční schopnost pozorovatele, ačkoli spouštění a zastavování stopek bylo prováděno dle doporučeného postupu v bodě nejvyšší rychlosti tělesa. Dalším možným zdrojem nepřesností byla nutnost využít pásové měřidlo pro měření vzdálenosti břitů na tyči (délka $2d$), které nebylo možné umístit do polohy vhodné pro přesné měření (tuto skutečnost reflektuje zvýšení systematické chyby z doporučené poloviny dílku na dílek celý).

Určitá chyba mohla vyplynout ze skutečnosti, že měřený předmět nebyl ve skutečnosti dokonalým kvádrem. Při srovnání vypočteného a naměřeného momentu setrvačnosti mohla odlišnost mezi hodnotami vzniknout skutečností, že při měření těleso nekmitalo podle své tělesové úhlopříčky, nýbrž stěnové úhlopříčky. Tato chyba však zjevně nebyla výrazná, rozdíl mezi oběma hodnotami pokrývá absolutní chyba měření respektive výpočtu.

Při ověřování Steinerovy věty vyšly hodnoty získané dvěma odlišnými metodami měření uspokojivě podobné, tento fakt je bohužel poněkud degradován relativně vysokou chybou měření způsobenou faktory zmíněnými výše.
\section*{Závěr}
Momenty setrvačnosti kvádru vzhledem k hlavní osám setrvačnosti:
$$ I_x = (3.64 \pm 0.09)\times \num{e-4} \ \si{\kilogram\metre\squared} $$
$$ I_y = (1.20 \pm 0.01)\times \num{e-3} \ \si{\kilogram\metre\squared} $$
$$ I_z = (1.50 \pm 0.01)\times \num{e-3} \ \si{\kilogram\metre\squared} $$

Složky jednotkového vektoru ve směru zadané obecné osy rotace kvádru v souřadné soustavě dané hlavními osami setrvačnosti:
$$ \nu_x = (886.5 \pm 0.5)\times \num{e-3} $$
$$ \nu_y = (443.5 \pm 0.4)\times \num{e-3} $$
$$ \nu_z = (131.7 \pm 0.1)\times \num{e-3} $$

Vypočítaný moment setrvačnosti kvádru vzhledem k zadané obecné ose rotace:
$$ I_{uv} = (5.49 \pm 0.09)\times \num{e-4} \ \si{\kilogram\metre\squared} $$

Měřený moment setrvačnosti kvádru vzhledem k zadané obecné ose rotace:
$$ I_{um} = (5.44 \pm 0.09)\times \num{e-4} \ \si{\kilogram\metre\squared} $$

Moment setrvačnosti tyče vůči ose procházející hmotným středem kolmo na tyč získaný Steinerovou větou:
$$ I_{tk} = (2.75 \pm 0.17)\times \num{e-3} \ \si{\kilogram\metre\squared} $$

Moment setrvačnosti tyče vůči ose procházející hmotným středem kolmo na tyč získaný metodou torzních kmitů:
$$ I_{tt} = (2.78 \pm 0.22)\times \num{e-3} \ \si{\kilogram\metre\squared} $$

\section*{Použitá literatura}
[1] Studijní text "Studium otáčení pevného tělesa", dostupné z\\ \url{http://physics.mff.cuni.cz/vyuka/zfp/_media/zadani/texty/txt_117.pdf}

[2] Doc. Mgr. Jakub Čížek, PhD.: prezentace Úvod do praktické fyziky, seminář 1, dostupné z \url{http://physics.mff.cuni.cz/kfnt/vyuka/upf/seminar1.pdf}

[3] Doc. Mgr. Jakub Čížek, PhD.: prezentace Úvod do praktické fyziky, seminář 9, dostupné z \url{http://physics.mff.cuni.cz/kfnt/vyuka/upf/seminar9.pdf}

[4] Doc. Mgr. Jakub Čížek, PhD.: prezentace Úvod do praktické fyziky, seminář 10, dostupné z \url{http://physics.mff.cuni.cz/kfnt/vyuka/upf/seminar10.pdf}

\end{document}
