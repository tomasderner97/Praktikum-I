\documentclass[protokol.tex]{subfiles}
\begin{document}
Hodnoty momentu setrvačnosti kola změřené dvěma různými způsoby sobě odpovídají těsně v rámci součtu jejich chyb. Za rozdíl mezi naměřenými výsledky může mimo jiné fakt, že na rozdíl od měření metodou otáčení metoda kyvů nepočítá s třením v závěsu kola. V případě metody kyvů mohly být chyby také způsobeny nízkou reakční schopností pozorovatele. 
Naopak při měření metodou otáčení se zdrojem nepřesností jeví měření poloměru souosých válců, některé z nich jsou příliž široké pro pohodlné měření posuvným měřidlem. Dále pak také do určité míry proměnlivá hmotnost závaží kvůli tíze provázku, který se s časem odvíjí z válce a přidává hmotnost, působící roztáčení kola.
\end{document}
