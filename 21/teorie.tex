\documentclass[protokol.tex]{subfiles}
\begin{document}

\subsection*{Měření tíhového zrychlení z doby kmitu kyvadla}
Tíhové zrychlení $g$ figuruje ve vztahu pro výpočet periody fyzického kyvadla
\begin{equation} \label{eq:T_fyz_kyv}
T = 2 \pi \sqrt{\frac{I}{mgd}}\left( 1 + \frac{1}{4} \sin^2 \frac{\alpha}{2} \right),
\end{equation}
kde $I$ moment setrvačnosti kyvadla vzhledem k ose otáčení, $m$ je hmotnost kyvadla, $g$ je  místní tíhové zrychlení, $d$ je  vzdálenost  těžiště od osy otáčení, $\alpha$ je maximální úhlová výchylka těžiště z rovnovážné polohy. Metodou fyzického kyvadla lze tedy tíhové zrychlení změřit.

Z důvodu zjednodušení výpočtů a měření aproximujeme fyzické kyvadlo kyvadlem matematickým, tedy hmotným bodem zavěšeným na nehmotném vlákně. Perioda tohoto kyvadla se pro malé výchylky spočte jako 
\begin{equation} \label{eq:T_mat_kyv}
T_M = 2 \pi \sqrt{\frac{l}{g}}
\end{equation}
pro délku závěsu $l$.
Ačkoliv fyzická realizace matematického kyvadla je nemožná, je možné se jí přiblížit zavěšením těžké kovové koule na tenký provázek a rozkmitáním s malým úhlem. Takto změříme tíhové zrychlení pomocí 
\begin{equation} \label{eq:g_mat_kyv}
g = \frac{4 \pi^2 l}{T_M^2}
\end{equation}

Jak bylo zmíněno výše, aproximace fyzického kyvadla matematickým není přesná, je tedy nutné spočíst systematickou chybu, kterou se tímto přiblížením dopouštíme. Vyjádříme tedy moment setrvačnosti fyzického kyvadla, změříme rozměry a spočteme tíhové zrychlení pomocí \eqref{eq:T_fyz_kyv}. Moment setrvačnosti homogenní koule o poloměru $R$ se spočítá jako 
\begin{equation} \label{eq:I_koule}
{I_k}_0 = \frac{2}{5} m_k R^2
\end{equation}
Moment setrvačnosti homogenní tyče délky $L$ vzhledem k ose kolmé k délce tyče a procházející jedním jejím koncem vyjádříme jako 
\begin{equation} \label{eq:I_tyce}
I_t = \frac{1}{3} m_t L^2
\end{equation}
Použitím Steineroy věty 
\begin{equation} \label{eq:steiner}
I = I_0 + ma^2,
\end{equation}
kde $m$ je hmotnost tělesa a $a$ vzdálenost os vypočítáme moment setrvačnosti kyvadla
\begin{equation} \label{eq:I_fyz_kyv}
I = \frac{2}{5} m_k R^2 + m_k \left( L + R \right) + \frac{1}{3} m_t L^2,
\end{equation}
považujeme-li provázek za homogenní tyč s délkou $L$. Považujeme-li úhlovou výchylku kyvadla $\alpha$ za dostatečně malou, můžeme vyjádřit tíhové zrychlení jako 
\begin{equation} \label{eq:g_fyz_kyv}
g = \frac{4 \pi^2 I}{m d T^2}.
\end{equation}

\subsection*{Reverzní kyvadlo}
Reverzní kyvadlo je fyzické kyvadlo, které kývá podle dvou rovnoběžných os. Jsou-li tyto dvě osy od sebe vzdáleny o redukovanou délku fyzického kyvadla $l_r$, kyvadlo kývá podle obou se stejnou periodou
\begin{equation} \label{eq:T_rev_kyv}
T_r = 2 \pi \sqrt{\frac{l_r}{g}}.
\end{equation}

V našem případě je vzdálenost os pevná, perioda kmitů se mění v závislosti na vzdálenosti těžké kovové čočky. Pro určení polohy čočky, při které kyvadlo kývá se stejnou periodou podle obou os, využijeme medotu grafické interpolace. Následně vypočítáme tíhové zrychlení z \eqref{eq:T_rev_kyv} jako
\begin{equation} \label{eq:g_rev_kyv}
g = \frac{4 \pi^2 l_r}{T_r^2}
\end{equation}

\subsection*{Těžiště fyzického kyvadla}
Jelikož reálný provázek není nehmotný, jeho hmotnost ovlivní polohu těžiště fyzikálního kyvadla. Těžiště obecného tělesa vypočítáme jako
\begin{equation} \label{eq:teziste_obecne}
\vec{r_s} = \frac{1}{M} \int\limits_{V} \vec{r}\rho(\vec{r})dV.
\end{equation}
Uvažujeme-li těžiště homogenní koule a homogenní tyče v jednorozměrném prostoru, dostaneme vzdálenost těžiště fyzického kyvadla od osy kmitání jako
\begin{equation} \label{eq:teziste_kyv}
l_s = \frac{m_k (L + R) + m_t \frac{L}{2}}{m_k + m_t}.
\end{equation}
\subsection*{Statistické vyhodnocení}
Průměrná hodnota naměřených veličin při $n$ měřeních je počítána podle vzorce aritmetického průměru 
$$ \overline{x} = \frac{1}{n} \sum\limits_{i=1}^n{x_i}.$$
Statistická chyba $\sigma_{stat}$ aritmetického průměru se získá ze vztahu
$$ \sigma_{stat} = \frac{\sqrt{\frac{1}{n-1} \sum\limits_{i=1}^n{(x_i - \overline{x})^2}}}{\sqrt{n}}. $$
Absolutní chyba je potom získána z $\sigma_{stat}$ a systematické chyby $\sigma_{sys}$ jako
$$ \sigma_{abs} = \sqrt{\sigma_{sys}^2 + \sigma_{stat}^2}$$

Chyba výpočtů se řídí zákonem přenosu chyb.

\subsection*{Pomůcky}
Posuvné měřidlo, pásové měřidlo, váhy, měřič času, provázek, kovová kulička, reverzní kyvadlo, závěsné zařízení
%\begin{comment}\end{comment}
\end{document}
