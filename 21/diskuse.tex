\documentclass[protokol.tex]{subfiles}
\begin{document}
Při výpočtu tíhového zrychlení pomocí matematického kyvadla bylo největším zdrojem chyb (kromě samotného přiblížení) měření délky závěsu, které bylo nepřesné kvůli špatné možnosti přiložit měřidlo, roli také hrála určitá elasticita provázku a také fakt, že provázek byl uvázán uzlem, který nemusel kmitat stejným způsobem, jako zbytek závěsu. Při výpočtu pomocí fyzického kyvadla k tomuto přistupuje také fakt, že kulička nebyla dokonalá homogenní koule. Dalším faktorem je rozdíl mezi předpokládanou a skutečnou polohou těžiště kyvadla. Díky automatickému měření času nepřibyla do měření chyba způsobená pomalou reakční dobou pozorovatele. 

Tak jako u matematického kyvadla, i u měření tíhového zrychlení pomocí reverzního kyvadla lze za výrazný zdroj chyby považovat měření délky mezi břity, ačkoli díky pevnému tělu kyvadla není tato chyba příliž velká. Tření břitů v závěsu a další disipativní síly se také mohly projevit.


\end{document}
