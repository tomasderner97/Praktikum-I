\documentclass[protokol.tex]{subfiles}
\begin{document}
Tato metoda měření povrchového napětí je poměrně náchylná na různé chyby měření. Z možných systematických chyb způsobených prostředím lze uvézt například vliv teploty okolního vzduchu, která jistě v průběhu měření nebyla zcela konstantní, nebo možnost znečištění kovových rámečků i při manipulaci pomocí pinzety.
Z chyb způsobených měřením fyzikálních veličin je vhodné uvézt problematické měření délky drátku $l$, kde kvůli realizaci rámečku je skutečná délka zkreslena pájenými spoji se stranami, a měření sil na torzních váhách, které je závislé na schopnosti experimentátora plynule otáčet řídicími kolečky.

V neposlední řadě se na nepřesnostech měření podílí fakt, že určení hodnoty síly, potřebné pro vyvážení rámečku pod hladinou kapaliny je velice subjektivní a může se lišit až o pět dílků stupnice. 
\end{document}
