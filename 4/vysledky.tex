\documentclass[protokol.tex]{subfiles}
\begin{document}
\begin{comment}
\begin{table}[H] \label{tab:podminky}
\centering
\setlength{\tabcolsep}{10pt}
\begin{tabular}{ccc}                                                    \toprule
Teplota                 &   Tlak                    &   Vlhkost     \\
$[\si{\degreeCelsius}]$ &   $[\si{\hecto\pascal}]$  &   [\% RH]     \\  \midrule
23,8                    &   1004,0                  &   21,4        \\  \bottomrule
\end{tabular}
\caption{Podmínky měření}
\end{table}

\begin{figure}[H]
\centering
\includegraphics[resolution=350]{plot/graf}
\caption{Graf závislosti napětí na relativním prodloužení}
\end{figure}
\end{comment}

Měření probíhalo při teplotě $24.5 \si{\celsius}$.

Délka drátu (vnitřní šířka rámečku) byla měřena posuvným měřidlem.
$$ l = (1.964 \pm 0.002) \times \num{e-2} \ \si{\metre} $$

Průměr drátu byl měřen mikrometrem na třech místech.
$$ r = (0.60 \pm 0.01) \times \num{e-3} \ \si{\metre} $$

V následující tabulce jsou uvedeny rozdíly hodnot naměřených torzními váhami $\Delta m$ a hodnoty $P_0$. Pro výpočty je využita přibližná hodnota tíhového zrychlení $g = 9.81 \si{\metre\per\second\squared}$.

\begin{table}[H] \label{tab:tab}
\centering
\setlength{\tabcolsep}{15pt}
\begin{tabular}{ccc}                                                                    \toprule
$c [\si{\percent}]$   &   $\Delta m [\si{\milli\gram}]$ &   $P_0 [\si{\newton}]$    \\  \midrule
100                   &   117                           &   0.00115                 \\
50                    &   151                           &   0.00148                 \\
25                    &   190                           &   0.00186                 \\
12.5                  &   230                           &   0.00226                 \\
6.25                  &   253                           &   0.00248                 \\
3.125                 &   255                           &   0.00250                 \\
1.5625                &   264                           &   0.00259                 \\
0                     &   297                           &   0.00291                 \\  \bottomrule
\end{tabular}
\caption{$\Delta m$ a $P_0$ v závislosti na koncentraci $c$}
\end{table}

\newpage

Následující graf zachycuje závislost povrchového napětí počítaného pomocí \eqref{eq:sigma}. Lineární regrese byla provedena pomocí křivky $y = ae^{bx} + c$.
\begin{figure}[H]
\centering
\includegraphics[resolution=350]{plot/out}
\caption{Graf závislosti povrchového napětí na koncentraci}
\end{figure}

\begin{table}[H] \label{tab:parametry}
\centering
\setlength{\tabcolsep}{10pt}
\begin{tabular}{ccccc}																						\toprule
	&	mosaz					&	měď					&	ocel				&	hliník				\\	\midrule
$k$	&	$0,0104 \pm 0,0001$		&	$0,0095 \pm 0,0001$ &	$0,0071 \pm 0,0001$	&	$0,0133 \pm 0,0002$	\\	
$q$	&	$-0,202 \pm 0,005 $		&	$-0,186 \pm 0,004 $ &	$-0,138 \pm 0,004 $	&	$-0,264 \pm 0,008 $	\\	\bottomrule
\end{tabular}
\caption{Parametry lineární regrese}
\end{table}
\end{document}
