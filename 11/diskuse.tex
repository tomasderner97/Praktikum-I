\documentclass[protokol.tex]{subfiles}
\begin{document}
Samotná surová data získaná experimentem jsou zatížena poměrně malou chybou díky uzpůsobení experimentu a prakticky vyloučením faktoru lidské chyby z procesu měření, ten byl totiž prakticky celý automatizován. Kvůli nedokonalému chování kovového vzorku při stlačování však dokonale neplatí Hookův zákon a průběh tak není ani do bodu úměrnosti zcela lineární.

Proto při tomto měření bylo nutné se na několika místech spoléhat na odhad, především při výběru lineárních částí průběhů funkcí. Z toho mohou být výsledné hodnoty zatíženy poměrně velkou systematickou chybou. Tyto chyby navíc nelze jednoduše přesně určit, jsou tedy samotné podrobeny stejné nepřesnosti způsobené subjektivním odhadem. 
\end{document}
