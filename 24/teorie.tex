\documentclass[protokol.tex]{subfiles}
\begin{document}
Teplotní objemovou roztažnost spočteme podle vztahu \cite{stud_text}
\begin{equation}
V = V_0 \left( 1 + \beta t \right),
\end{equation}
kde $V$ je objem při teplotě $t$, $V_0$ objem měřený při teplotě $0 \si{\celsius}$ a $\beta$ součinitel objemové roztažnosti \cite{stud_text}
\begin{equation} \label{eq:souc_objem}
\beta = \frac{1}{V_0} \left( \frac{\partial V}{\partial t} \right)_p.
\end{equation}

Pro tělesa, jejichž jeden rozměr je výrazně větší než ostatní, zavádíme délkovou roztažnost \cite{stud_text}
\begin{equation}
l = l_0 \left( 1 + \alpha t \right)
\end{equation}
vzhledem k délce tělesa $l$ při teplotě $t$. Součinitel délkové roztažnosti vypočítáme jako \cite{stud_text}
\begin{equation} \label{eq:souc_delka}
\alpha = \frac{1}{l_0} \left( \frac{\partial l}{\partial t} \right)_p.
\end{equation}
V případě izotropních těles platí \cite{stud_text}
\begin{equation} \label{eq:vztah_ab}
\beta = 3 \alpha.
\end{equation}

\subsection*{Statistické vyhodnocení}
Průměrná hodnota naměřených veličin při $n$ měřeních je počítána podle vzorce aritmetického průměru 
\cite{cizek_10}
$$ \overline{x} = \frac{1}{n} \sum\limits_{i=1}^n{x_i}.$$
Statistická chyba $\sigma_{stat}$ aritmetického průměru se získá ze vztahu \cite{cizek_10}
$$ \sigma_{stat} = \frac{\sqrt{\frac{1}{n-1} \sum\limits_{i=1}^n{(x_i - \overline{x})^2}}}{\sqrt{n}}. $$
Absolutní chyba je potom získána z $\sigma_{stat}$ a chyby měřidla $\sigma_{\text{měř}}$ jako \cite{cizek_1}
$$ \sigma_{abs} = \sqrt{\sigma_{\text{\textit{měř}}}^2 + \sigma_{stat}^2}$$

Chyba výpočtů se řídí zákonem přenosu chyb \cite{cizek_9}, lineární regrese podle metody nejmenších čtverců 
\cite{cizek_11}.

\subsection*{Pomůcky}
Dilatometr (držák tyče a indikátorové hodinky), pásové měřidlo, teploměr, nádrž s vodou, čerpadlo, ohřívač vody, tyče ze čtyř materiálů (mosaz, měď, ocel, hliník)
\end{document}
