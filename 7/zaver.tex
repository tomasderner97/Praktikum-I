\documentclass[protokol.tex]{subfiles}
\begin{document}
Doba kmitu dvou stejných nevázaných fyzických kyvadel je 
$$ T_0 = (1,88 \pm 0,01) \ \si{\second}. $$

Změřené doby kmitů jsou
$$T_{sl_1} = (1,87 \pm 0,01) \ \si{\second},$$
$$T_{sl_2} = (1,79 \pm 0,01) \ \si{\second},$$
$$T_{sl_3} = (1,84 \pm 0,01) \ \si{\second},$$
$$T_{sl_4} = (41,48 \pm 0,32) \ \si{\second}$$
$$T_{si_1} = (1,87 \pm 0,01) \ \si{\second},$$
$$T_{si_2} = (1,73 \pm 0,01) \ \si{\second},$$
$$T_{si_3} = (1,81 \pm 0,01) \ \si{\second},$$
$$T_{si_4} = (21,85 \pm 0,19) \ \si{\second}.$$

Kruhové frekvence odpovídající jednotlivým dobám kmitů jsou
$$ \omega_{sl_1} = (3,361 \pm 0,016) \ \si{\per\second}, $$
$$ \omega_{sl_2} = (3,507 \pm 0,019) \ \si{\per\second}, $$
$$ \omega_{sl_3} = (3,415 \pm 0,019) \ \si{\per\second}, $$
$$ \omega_{sl_4} = (0,076 \pm 0,001) \ \si{\per\second}, $$
$$ \omega_{si_1} = (3,361 \pm 0,016) \ \si{\per\second}, $$
$$ \omega_{si_2} = (3,637 \pm 0,019) \ \si{\per\second}, $$
$$ \omega_{si_3} = (3,475 \pm 0,025) \ \si{\per\second}, $$
$$ \omega_{si_4} = (0,144 \pm 0,001) \ \si{\per\second}. $$

Hodnoty $\omega_3$ a $\omega_4$ vypočtené z \eqref{eq:omega_3} a \eqref{eq:omega_4} se neliší o více než chybu měření.

Stupně vazby pro slabou a silnou pružinu jsou
$$ \kappa_{sl} = (0,042 \pm 0,007) $$
$$ \kappa_{si} = (0,079 \pm 0,007). $$

\end{document}
