\documentclass[protokol.tex]{subfiles}
\begin{document}
\begin{enumerate}
\item Změřte dobu kmitu $T_0$ dvou stejných nevázaných fyzických kyvadel.
\item Změřte doby kmitů $T_i$ dvou stejných fyzických kyvadel vázaných slabou pružnou vazbou vypouštěných z klidu při počátečních podmínkách:
\begin{enumerate}
\item $y_1 = y_2 = B$... doba kmitu $T_1$
\item $y_1 = -y_2 = B$... doba kmitu $T_2$
\item $y_1 = 0, y_2 = B$
\begin{enumerate}
\item doba kmitu $T_3$
\item doba $\frac{T_s}{2}$, za kterou dojde k maximální výměně energie mezi kyvadly
\end{enumerate}
\end{enumerate}
\item Vypočtěte kruhové frekvence $\omega_0$, $\omega_1$, $\omega_2$, $\omega_3$, $\omega_4$ odpovídající dobám $T_0$, $T_1$, $T_2$, $T_3$ a $T_s$, ověřte měřením platnost vztahů odvozených pro $\omega_3$ a $\omega_4$.
\item Vypočtěte stupeň vazby $\kappa$.
\item Pro jednu pružinu změřte závislost stupně vazby na vzdálenosti zavěšení pružiny od uložení závěsu kyvadla a graficky znázorněte.
\end{enumerate}
\end{document}
