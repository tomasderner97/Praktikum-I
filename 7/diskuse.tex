\documentclass[protokol.tex]{subfiles}
\begin{document}
V úkolu číslo 1 nebyly měřeny zvlášť periody obou fyzických kyvadel, nýbrž byla tato kyvadla kalibrována tak, aby opticky působila ve fázi i po několika kmitáních. Tento fakt však nelze považovat za výrazný zdroj chyb. V tuto dobu se však také objevila skutečnost, že druhé z kyvadel mělo ve svém závěsu z technických důvodů výrazně vyšší tření, a tedy bylo také poznatelně více tlumeno. Určitá chyba se tedy dá připsat tomuto tlumení, obzvlášť v případě měření $T_s$. Tuto skutečost dokládá přiložený pracovní graf, na kterém je patrné výrazné prodloužení periody kmitů obalové křivky.
Největším zdrojem chyb bylo nesporně poněkud problematické zpracování výsledků měření na počítači, při kterém se pracovalo z části odhadem, především pak při měření $T_s$, kdy nebylo možné přesně rozeznat, v kterou chvíli dosáhla obalové křivka vlnění minima či maxima. I přes tento fakt však hodnoty, vypočítané z týchž naměřených dat různámi způsoby, sobě až na chybu měření odpovídají.
Před začátkem měření pátého úkolu se podařilo eliminovat již zmíňěný problém se zvýšeným třením jednoho z kyvadel, tento fakt se tedy již nepodílel na chybě měření.
\end{document}
